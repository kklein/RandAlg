
\documentclass[a4paper,german]{article}

\usepackage{amsmath}
\usepackage{amssymb}
\usepackage{amsthm}
\usepackage{bbm}

\newcommand\loe{\mathrel{\overset{\makebox[0pt]{\mbox{\normalfont\tiny\sffamily LOE}}}{=}}}
\newcommand\ub{\mathrel{\overset{\makebox[0pt]{\mbox{\normalfont\tiny\sffamily UB}}}{\leq}}}

%This is how you can create a "claim"-environment (or a lemma/Theorem/definition etc environment)
\newtheorem{claim}{Claim}

%This is how you can create a command of your own, e.g. to simplify the usage signs you often use.
\newcommand{\E}{\mathbb{E}}

%Titlepage:
\title{Randomized Algorithms and Probabilistic Methods: Graded Homework 1}
\author{ Kevin Klein, collaborators: Ramon Braunwarth, Fabian Gerhardt}
\date{October 23rd 2017}


\begin{document}
\maketitle

\section*{Exercise 1}
\section*{Exercise 2}
\section*{Exercise 3}
\section*{Exercise 4}
TODOs: Stirling formula, exact size (i.e. rounding error), prove min
\begin{enumerate}
\item
First, we'll bound the probability of a given subset of size \(n/2\) not containing a triangle. In this subset, there are \( {n/2 \choose 3}\) many possible triangles, over which we assume an arbitrary order. We define 
\begin{align*} 
T_i &= \mathbb{I} [\text{triangle } i \text{ is present in subset } 1] = p^3 \\
T &= \sum_{i \in {n/2 \choose 3}} T_i \\
\lambda &= \E[T] \loe \sum_{i \in {n/2 \choose 3}} \E[T_i]
\end{align*}
In order to calculate \(\Delta\), we notice that \(I_i = \{e_{i1}, e_{i2}, e_{i3} \}\), where \(e_{ik}\) are the edges constructing the triangle \(i\). We proceed by formulating a case distinction on the number of shared edges between two triangles \(i\) and \(j\).
\begin{enumerate}
\item If  they share no edges, \(I_i \cap I_j = \emptyset \).
\item If they share 1 edge, the probability of both being present is equal to the probability of those 5 eges being selected, i.e. \(\Pr[T_i \wedge T_j] = p^5\).
\item If they share 2 edges, the third edge needs to be the same for both triangles as there is at most a single edge connecting two nodes. Hence \(i = j\).
\item If they share 3 edges, \(i = j\) holds trivially. 
\end{enumerate}
We can conclude that the only relevant case is the second. It remains to inspect how frequently this case can occur. In a given subset, there are \( {n/2 \choose 3} \) possible triangles, of which each can share each of its 3 edges. When sharing an edge, the second triangle can be 'completed' by \( (n/2 -3\) remaining nodes. Summing up, we have:

\begin{align*}
\Delta = {n/2 \choose 3}3(n/2 -3) p^5
\end{align*}
Janson's equality tells us:
\begin{align*}
\Pr[T = 0] = e^{- min\{\lambda, \lambda^2 / \Delta\} /4} 
\end{align*}

We will now use this knowledge about the probability of a given subset containing no triangle to estimate the probability of each subset containing a triangle. We observe that there are \( {n \choose n/2}\) ways to partition the nodes into subsets of size \(n/2\). 
\begin{align*}
\Pr[\text{each subset contains a triangle}] &= \Pr[ \bigwedge_i \text{subset } i \text{ contains a triangle}] \\
&= 1 - \Pr[\bigvee_i \text{subset } i \text{ contains no triangle}]
\end{align*}
We look into the latter probability.
\begin{align*}
&\Pr[\bigvee_i \text{subset } i \text{ contains no triangle}]  \\
&\ub \sum_i \Pr[\text{subset } i \text{ contains no triangle}] \\
&\leq  {n \choose n/2} \Pr[\text{subset } 1 \text{ contains no triangle}] \text{   (all subsets symmetric)} \\
&\leq \frac{2^n}{\sqrt n}  \Pr[\text{subset } 1 \text{ contains no triangle}] \text{   (Stirling approximation)} \\
&\leq \frac{2^n}{\sqrt n}  \Pr[T = 0]\\
\end{align*}

\item
\end{enumerate}
\end{document}
