
\documentclass[a4paper,german]{article}

\usepackage{amsmath}
\usepackage{amssymb}
\usepackage{amsthm}
\usepackage{bbm}


%This is how you can create a "claim"-environment (or a lemma/Theorem/definition etc environment)
\newtheorem{claim}{Claim}

%This is how you can create a command of your own, e.g. to simplify the usage signs you often use.
\newcommand{\E}{\mathbb{E}}

%Titlepage:
\title{Randomized Algorithms and Probabilistic Methods: Graded Homework 1}
\author{ Kevin Klein, list of collaborators}
\date{October 23rd 2017}


\begin{document}
\maketitle

\section*{Exercise 1}
Let's define \(C = \bigcup_{v} C_v\). If we can find a partition \(C = C_A \cup C_B\) such that \(\forall v \in A: C_v \cap C_A \neq \emptyset \) and \(\forall v \in B: C_v \cap C_B \neq \emptyset \), we have found a trivial admissable coloring. The admissable coloring could be constructed by coloring each node with an arbitrary color from said intersection.
We now want to show that, given the minimal length of \(C_v\), a random partitioning can lead to an admissable coloring. \\

We assign each \(c \in C\) to either \(C_A\) or \(C_B\) u.a.r. This is a partition as every color is assigned to exactly one set. 
When inspecting node \(v \in A\), we define \emph{success} to be the existence of \(c \in C_V \cap C_A \neq \emptyset\). The case for which \(v \in B\) follows by symmetry. Success for all nodes implies that each node can chose a color which is not chosen in the class containing its neighboring nodes, i.e. there is an admissable coloring. 

\begin{align*} X_i &= \text{\emph{success} for node } i  \\
&= \mathbbm{1}[C_i \cap C_{class(i)} \neq \emptyset] \\
\Pr[X_i] &= 1 - \Pr[\text{all } c \in C_i \text{ have been mapped to } C - C_{class(i)}] \\
 &= 1 - \frac{1}{{2}^{|C_i|}} \\
 X &= \text{\#\emph{successes} among all } 2n \text{ nodes}\\
 &= \sum_{i=1}^{2n} X_i \\
 \mathbb{E}[X] &= \mathbb{E}[\sum_{i=1}^{2n} X_i ]\\
 &= \sum_{i=1}^{2n} \mathbb{E}[X_i ] \text{ by linearity of expectation} \\
 &=  \sum_{i=1}^{2n} \Pr[X_i ] \\
&=  \sum_{i=1}^{2n} (1 - \frac{1}{{2}^{|C_i|}}) \\
\end{align*}

We can now make use of our knowledge about the size of each \(C_v\).

\begin{align*}
|C_v| > log_2n + 1 &\Rightarrow \frac{1}{{2}^{|C_v|}} < \frac{1}{{2}^{log_2n + 1}} = \frac{1}{2n}\\
&\Rightarrow 1 - \frac{1}{{2}^{|C_v|}} > 1 - \frac{1}{2n} \\
\end{align*}

We can now use this this bound to reformulate the expectation.

\begin{align*} \mathbb{E}[X] &> \sum_{i=1}^{2n}  (1 - \frac{1}{2n})  \\
&> 2n(1 - \frac{1}{2n} ) \\
&> 2n - 1 \\
\end{align*}
By the probabilistic method, there has to be at least one realistion of \(X\) attaining value at least \(2n\).  The latter implies \emph{success} for each node, which, as we have argues, implies the existence of an admissable coloring.

\section*{Exercise 2}


\end{document}
